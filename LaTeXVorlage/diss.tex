%%%%%%%%%%%%%%%%%%%%%%%%%%%%%%%%%%%%%%%%%
%%            LMU-Vorlage              %%
%%                                     %%
%%         zur Erstellung einer        %%
%%   Dissertation mit pdflatex/latex   %%
%%                                     %%
%%  (2002) Robert Dahlke               %%
%%         & Sigmund Stintzing         %%
%%%%%%%%%%%%%%%%%%%%%%%%%%%%%%%%%%%%%%%%%

\documentclass[12pt]{book}


%%%%%%%%%%%%%%%%%%%%%%%%%%%%
%%   Zusaetzliche Pakete  %%
%%%%%%%%%%%%%%%%%%%%%%%%%%%%

\usepackage{a4wide}
\usepackage{fancyhdr}
\usepackage{graphicx}
\usepackage{german}
\usepackage[bookmarks]{hyperref}


%%%%%%%%%%%%%%%%%%%%%%%%%%%%%%
%% Definition der Kopfzeile %%
%%%%%%%%%%%%%%%%%%%%%%%%%%%%%%

\pagestyle{fancyplain}
\renewcommand{\chaptermark}[1]%
         {\markboth{\thechapter.\ #1}{}}
\renewcommand{\sectionmark}[1]%
         {\markright{\thesection\ #1}}
\lhead[\fancyplain{}{\bfseries\thepage}]%
    {\fancyplain{}{\bfseries\rightmark}}
\rhead[\fancyplain{}{\bfseries\leftmark}]%
    {\fancyplain{}{\bfseries\thepage}}
\cfoot{}


%%%%%%%%%%%%%%%%%%%%%%%%%%%%%%%%%%%%%%%%%%%%%%%%%%%%%
%%  Definition des Deckblattes und der Titelseite  %%
%%%%%%%%%%%%%%%%%%%%%%%%%%%%%%%%%%%%%%%%%%%%%%%%%%%%%

\newcommand{\LMUTitle}[9]{
  \thispagestyle{empty}
  \vspace*{\stretch{1}}
  {\parindent0cm
   \rule{\linewidth}{.7ex}}
  \begin{flushright}

    \vspace*{\stretch{1}}
    \sffamily\bfseries\Huge
    #1\\
    \vspace*{\stretch{1}}
    \sffamily\bfseries\large
    #2
    \vspace*{\stretch{1}}
  \end{flushright}
  \rule{\linewidth}{.7ex}
  \vspace*{\stretch{5}}
  \begin{center}
    \includegraphics[width=2in]{siegel}
  \end{center}
  \vspace*{\stretch{1}}
  \begin{center}\sffamily\LARGE{#5}\end{center}
  \newpage
  \thispagestyle{empty}

  \cleardoublepage
  \thispagestyle{empty}

  \vspace*{\stretch{1}}
  {\parindent0cm
  \rule{\linewidth}{.7ex}}
  \begin{flushright}
    \vspace*{\stretch{1}}
    \sffamily\bfseries\Huge
    #1\\
    \vspace*{\stretch{1}}
    \sffamily\bfseries\large
    #2
    \vspace*{\stretch{1}}
  \end{flushright}
  \rule{\linewidth}{.7ex}

  \vspace*{\stretch{3}}
  \begin{center}
    \Large Dissertation\\
    \Large an der #4\\
    \Large der Ludwig--Maximilians--Universit"at\\
    \Large M\"unchen\\
    \vspace*{\stretch{1}}
    \Large vorgelegt von\\
    \Large #2\\
    \Large aus #3\\
    \vspace*{\stretch{2}}
    \Large M\"unchen, den #6
  \end{center}

  \newpage
  \thispagestyle{empty}

  \vspace*{\stretch{1}}

  \begin{flushleft}
    \large Erstgutachter:  #7 \\[1mm]
    \large Zweitgutachter: #8 \\[1mm]
    \large Tag der m\"undlichen Pr\"ufung: #9\\
  \end{flushleft}

  \cleardoublepage
}




%%%%%%%%%%%%%%%%%%%%%%%%%%%%
%%  Beginn des Dokuments  %%
%%%%%%%%%%%%%%%%%%%%%%%%%%%%

\begin{document}


  \frontmatter


  \LMUTitle
      {Dies ist Titel der Dissertation \\
       eventuell mit Untertitel}               % Titel der Arbeit
      {Vorname Nachname}                       % Vor- und Nachname des Autors
      {Geburtsort}                             % Geburtsort des Autors
      {Fakult"atsname}                         % Name der Fakultaet
      {M"unchen 2002}                          % Ort und Jahr der Erstellung
      {Abgabedatum}                            % Tag der Abgabe
      {Erstgutachter}                          % Name des Erstgutachters
      {Zweitgutachter}                         % Name des Zweitgutachters
      {Pr"ufungsdatum}                         % Datum der muendlichen Pruefung


  \tableofcontents
  \markboth{Inhaltsverzeichnis}{Inhaltsverzeichnis}


  \listoffigures
  \markboth{Abbildungsverzeichnis}{Abbildungsverzeichnis}


  \listoftables
  \markboth{Tabellenverzeichnis}{Tabellenverzeichnis}
  \cleardoublepage


  \markboth{Zusammenfassung}{Zusammenfassung}
  \include{zusammenfassung}


  \mainmatter\setcounter{page}{1}
  \chapter{Das erste Kapitel}

Dies ist das erste Kapitel der Dissertation. Wir zitieren
\cite{Autor1:02}.


\section{Abschnitt Eins}

Dies ist der erste Abschnitt im ersten Kapitel.


%%%%%%%%%%%%%%%%%%%%%%%%%%%%%%
%%  Einbinden einer Grafik  %%
%%%%%%%%%%%%%%%%%%%%%%%%%%%%%%

\begin{figure}[htb]
  \centering
  \includegraphics[scale=0.5]{siegel}
  \caption[Kurzform f"ur das Abbildungsverzeichnis]{Dies ist die Erkl"arung zum Bild.}
\end{figure}


\section[Kurzform]{Abschnitt Zwei}

Jetzt kommt der zweite Abschnitt im ersten Kapitel
mit dem Zitat \cite{Autor2:01}.

  \chapter{Das zweite Kapitel}

Dies ist das zweite Kapitel der Dissertation.


\section{Abschnitt Eins}

Dies ist der erste Abschnitt im zweiten Kapitel.


%%%%%%%%%%%%%%%%%%%%%%%%%%%%%%%%%%
%%  Beispiel fuer eine Tabelle  %%
%%%%%%%%%%%%%%%%%%%%%%%%%%%%%%%%%%

\begin{table}[htb]
\centering
\begin{tabular}{l|l}
Erste Spalte & Zweite Spalte \\ \hline
Eintrag & Eintrag
\end{tabular}
 \caption[Kurzform f"ur das Tabellenverzeichnis]{Dies ist die Erkl"arung zur Tabelle.}
\end{table}


\section{Abschnitt Zwei}

Jetzt kommt der zweite Abschnitt im zweiten Kapitel.

  %\include{kap_03}
  %\include{kap_04} usw.
  \include{anhang}


  \backmatter
  \include{bibliographie}
  \markboth{}{}


  \include{danksagung}


  \chapter*{Lebenslauf}

Sigmund Stintzing

\vspace*{2.0cm}

\begin{tabular}{ll}

Geburtsdatum & Geburt in Geburtsort \\[1.5ex]
Schulzeit & Besuch der Schule in Ort \\[1.5ex]
 ... & ...
\end{tabular}



\end{document}
